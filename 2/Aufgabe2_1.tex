\documentclass{article}

\usepackage{enumerate}

\begin{document}

Matrikelnummer: 776139
\\
\section{Aufgabe 2.1}
\begin{enumerate}[a)]
	\item
	\begin{itemize}
		\item
		Vorname: ([\textbackslash{}p\{Lu\}]+[\textbackslash{}p\{Ll\}]*\textbackslash{}.?(-\textbar{}\textbackslash{} )?)+\\
		\item
		Nachname: [\textbackslash{}p\{Lu\}]\{1\}[\textbackslash{}p\{Ll\}]+
		\item
		Strasse: ([\textbackslash{}p\{Lu\}]?[\textbackslash{}p\{Ll\}]+\textbackslash{}.?(-\textbar{}\textbackslash{} )?)+\\
		\item
		Nummer: [0-9]\{1,3\}[A-Za-z]?\\
		\item
		Postleitzahl: [0-9]\{5\}
		\item
		Ort: ([\textbackslash{}p\{Lu\}]?[\textbackslash{}p\{Ll\}]+\textbackslash{}.?(-\textbar{}\textbackslash{} )?)+
	\end{itemize}
	\item
	\begin{itemize}
		\item
		Adresszusatz: c/o\textbackslash{} .+
		\item
		Muss nicht erweitert werden.
		\item
		Nachname: [\textbackslash{}p\{Lu\}][\textbackslash{}p\{Ll\}]+(-\textbackslash{}p\{Lu\}[\textbackslash{}p\{Ll\}]+)?
	\end{itemize}
	Gesamtausdruck kombiniert (wobei \textless{}xxxx\textgreater{} stellvertetend f\"ur den regex-Ausdruck xxxx steht):\\
	\textless{}Vorname\textgreater{}[,;\textbackslash{} ]\textless{}Nachname\textgreater{}(\textbackslash{}n\textbar{}\textbackslash{}r\textbackslash{}n)\textless{}Strasse\textgreater{}[,;-\textbackslash{} ]\textless{}Nummer\textgreater{}(\textbackslash{}n\textbar{}\textbackslash{}r\textbackslash{}n)\textless{}Postleitzahl\textgreater{}[,;-\textbackslash{} ]\textless{}Ort\textgreater{}(\textbackslash{}n\textbar{}\textbackslash{}r\textbackslash{}n)\textless{}Adresszusatz\textgreater{}\\
	\\
\end{enumerate}


\end{document}
